\section{Antecedentes}

Una definición de una estructura FHIR se hace por medio de un recurso StructureDefinition \cite{FHIRStructureDefinition}. Estas estructuras son usadas para describir los recursos, y se representan como listas planas de elementos. Cada elemento incluye una ruta, una cardinalidad y un tipo de dato \cite{FHIRElementDefinition}. La ruta es la propiedad más importante de la definición del elemento. Esta ruta localiza el elemento en una jerarquía definida dentro de la estructura. En algunos casos, los recursos StructureDefinition pueden usar conjuntos de valores para especificar el contenido de los elementos codificados.

FHIR define recursos StructureDefinition para una serie de diferentes tipos de recursos. Estos tipos de recursos son usados para representar conceptos administrativos y clínicos \cite{FHIRResourceList}. Por cada tipo de recurso, FHIR define un conjunto de elementos de datos diferente \cite{FHIRResource}. Cada uno de estos elementos de datos usa un tipo de dato. FHIR especifica diferentes categorías de tipos de datos \cite{FHIRDataTypes}. Entre los cuales, se encuentra los tipos de datos primitivos que permiten un solo valor. Las demás categorías son tipos de datos complejos que tienen elementos hijos.

En openEHR, un arquetipo constituye una encapsulación de un conjunto de puntos de datos pertencientes a un contenido de dominio, expresado en términos de restricciones del modelo de información de referencia \cite{openEHRArchetype}. La definición de un arquetipo consiste en capas alternativas de nodos de restricción de objeto y atributo \cite{openEHRAOM}. Los diferentes tipos de nodos son: nodos de restricción de primitivos que restringen tipos de datos primitivos, nodos que representan referencias a otros nodos, nodos de referencia de restricción que hacen referencia a una restricción de texto en la parte de vinculación de restricción de la terminología del arquetipo, y nodos de restricción de arquetipo que representan restricciones en otros arquetipos permitidos para aparecer en un punto dado. Estas restricciones definen que configuraciones de instancias de clase de modelo de referencia se consideran conformes al arquetipo. Por ejemplo, ciertas configuraciones de las clases PARTY, ADDRESS, CLUSTER y ELEMENT pueden definirse por un arquetipo Person como estructuras permitidas para personas con identidad, contactos y direcciones. Las estructuras jerárquicas objeto y atributo repetidas de un arquetipo proporciona la base para usar rutas para hacer referencia a cualquier nodo en un arquetipo.

Los arquetipos son expresados en un genérico Lenguaje de Definición de Arquetipo \cite{openEHRADL} (ADL por sus siglas en inglés). ADL utiliza tres sintaxis: cADL (forma d restricción de ADL), ODIN (Notación de instancia de datos de objeto) y una versión de lógica de predicado de primer orden (FOPL por sus siglas en inglés). La sintaxis de cADL se utiliza para expresar la definición de los arquetipos, mientras que la sintaxis de ODIN se usar para expresar datos que aparecen en las secciones de idioma, descripción, terminología y revisión histórica de un arquetipo.

FHIR y openEHR comparten cierta similitud. Los recursos de FHIR y los arquetipos de openEHR definen patrones reutilizables para la descripción precisa de la información clínica \cite{Bosca15}. Sin embargo, los trabajos de colaboración entre las comunidades de FHIR y openEHR \cite{Collaboration} no consiguieron generar recursos y arquetipos con un contenido coincidente y clínicamente verificable. Una de las causas son los principios de diseño diferentes utilizados por ambas comunidades. Siendo la principal diferencia que los arquetipos esperan representar la mayoría del contenido clínico, mientras que los recursos solo contienen la información clínica utilizada más común.
