\section{Antecedentes}

Maldonado et. al. proponen una formalización de la definición de la sección de arquetipos basada en tipos sobre una estructura de árbol. Esta formalización es independiente de la sintaxis especificada en ADL y se soporta en un sistema de tipos que modela las restricciones estructurales especificadas en los arquetipos \cite{Maldonado09}. La base de su formalización es la Lista de Multiplicidad Restringida (CML por sus siglas en inglés), la cual es un lenguaje de definición que abstrae arquetipos por tipos. Parte de este trabajo extiende esta propuesta agregando la sección de terminología y el uso de identificadores que se pueden utilizar directamente para definir reglas de mapeo entre los recursos FHIR y los datos en los repositorios openEHR.

Boscá et. al. mencionan que existe una similitud entre FHIR y enfoques como el utilizado en openEHR. Tanto los recursos de FHIR y los arquetipos de openEHR definen patrones reutilizables para la descripción precisa de la información clínica. Siendo la principal diferencia que los arquetipos esperan representar la mayoría del contenido clínico, mientras que los recursos solo contienen la información clínica utilizada más común \cite{Bosca15}. FHIR proporciona la capacidad de manejar información adicional por medio de extensiones. Esta similitud da origen al método de este trabajo.

Existió colaboración entre las comunidades de FHIR y openEHR para revisión en conjunto de recursos y arquetipos de contenido equivalente \cite{Collaboration}. La intención de la revisión fue si incluso con una representación diferente de los modelos de información es posible encontrar un mapeo entre los dos enfoques.
