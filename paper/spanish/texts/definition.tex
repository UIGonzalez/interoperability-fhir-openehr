\subsubsection{Etapa de definición}

Según las definiciones de un esquema openEHR, se crea un arquetipo openEHR usando Archetype Definition Language (ADL) \cite{openEHRADL}.

Como identificador del nuevo arquetipo openEHR se usa un nombre de espacio fhir y el nombre del recurso FHIR que el esquema openEHR modela. Por ejemplo, el identificador del arquetipo openEHR del recurso SimplePatient es:

\begin{lstlisting}
fhir::openEHR-EHR-CLUSTER.SimplePatient.v1.0.0
\end{lstlisting}

Para la sección de definición del nuevo arquetipo openEHR, los tipos del esquema openEHR abstraídos de tipos primitivos de FHIR son expresados usando la estructura ELEMENT, y los demás tipos del esquema openEHR son expresados usando la estructura CLUSTER. Ambas estructuras pertenecen del Modelo de Información de Estructuras de Datos de openEHR \cite{openEHRDataStructures}. Por ejemplo, la sección de definición del arquetipo openEHR del recurso SimplePatient usa las clases openEHR CLUSTER y openEHR ELEMENT para los tipos \(T_{SimplePatient}\) y \(T_{SimplePatient.gender}\) respectivamente:

\begin{lstlisting}[mathescape=true]
CLUSTER[id1] $\in$ {
  items $\in$ {
    ELEMENT[id2] occurrences $\in$ {0..1} $\in$ {
      value $\in$ {
        DV_TEXT[id3]
      }
    }
  }
}
\end{lstlisting}

Las definiciones de vinculación se agrega en la sección de vinculación de términos del nuevo arquetipo openEHR. Por ejemplo, la definición de vinculación del tipo \(T_{SimplePatient.gender.DV\_TEXT.value}\) del recurso SimplePatient se transcribe en la sección de vinculación de términos de su arquetipo openEHR equivalente:

\begin{lstlisting}
term_bindings = <
  [``fhir"] = <
    [``id3"] = <
http://hl7.org/ValueSet/administrative-gender
    >
  >
>
\end{lstlisting}

Un arquetipo openEHR permite la formación de rutas ADL, los cuales sirven para identificar elementos de los arquetipos openEHR \cite{openEHRArchitecture}. En esta etapa, se crea las rutas ADL del nuevo arquetipo openEHR y se relaciona con las rutas de los elementos de su recurso FHIR equivalente. Considerando el recuso de SimplePatient, la relación que se establece es:

\begin{lstlisting}[mathescape=true]
SimplePatient.gender $\Leftrightarrow$ /items[id2]/value[id3]
\end{lstlisting}
