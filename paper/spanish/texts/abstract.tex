\begin{abstract}

\textit{Propósito:} Crear arquetipos de integración openEHR a partir de definiciones de recursos FHIR que permitan importar datos de recursos FHIR dentro de sistemas openEHR. La interoperabilidad de los sistemas de información que dan soporte al proceso asistencial es un factor crucial para una adecuada continuidad de cuidados.

\textit{Métodos:} Un prerrequisito para convertir instancias de recursos FHIR en instancias de arquetipos de integración openEHR es encontrar equivalencias entre los tipos de datos FHIR y openEHR. Este trabajo propone una metodología para encontrar dichas equivalencias de manera manual y un proceso de creación automático de arquetipos de integración.

\textit{Resultados:} Con base en las equivalencias, se diseñó un proceso de creación automático basado en un sistema de tipos con soporte para vinculaciones a terminologías. A partir del diseño, se implementó el proceso descripto en el lenguaje Python.

\textit{Conclusiones:} La implementación del proceso automático posibilita la creación de arquetipos de integración openEHR a partir de recursos FHIR en menor tiempo que el proceso manual, y asigna como responsables de la importación de datos de recursos FHIR dentro de sistemas openEHR a los especialistas de la tecnología de información. La compatibilidad entre los tipos de datos FHIR y openEHR ayuda a que se pueda convertir y procesar correctamente los datos de instancias de recursos FHIR en instancias de arquetipos de integración openEHR.

\end{abstract}
