\section{Métodos}

Una forma de encontrar correspondencia entre un recurso FHIR y un arquetipo openEHR es encontrar relaciones entre elementos de datos del recurso FHIR y puntos de datos del arquetipo openEHR. Para encontrar las relaciones se crea un nuevo arquetipo openEHR a partir de la definición del recurso FHIR existente.

La creación del arquetipo openEHR a partir de la definición del recurso FHIR utilizando equivalencias entre tipos de datos puede ser sencilla. El arquetipo openEHR se puede crear de forma manual dentro de un editor de arquetipos como los citados en \cite{openEHRModellingTools}. El arquetipo openEHR se modela con la misma estructura del recurso FHIR. Cada tipo de dato de FHIR de la estructura se sustituye por su equivalente tipo de dato de openEHR. En este trabajo, se presenta una alternativa diferente consistente en la creación a través de un proceso automatizado. Las ventajas de la creación automática son un tiempo de creación y un riesgo de introducir errores por factor humano menores a los de la creación manual.

\subsection{Equivalencias entre tipos de datos}

Una equivalencia entre un tipo de dato primitivo de FHIR y un tipo de dato de openEHR existe si se cumplen las condiciones de:

\begin{enumerate}
  \item el dominio de valores del tipo de dato de FHIR puede ser almacenado dentro de uno de los atributos del tipo de dato de openEHR;
  \item el uso del tipo de dato de openEHR para el cual fue diseñado se conserva.
\end{enumerate}

Como un ejemplo, el tipo de FHIR boolean y el tipo de openEHR DV\_BOOLEAN reúnen ambas condiciones, por lo tanto, son equivalentes:
\begin{enumerate}
  \item los valores true y false del dominio de valores del tipo boolean pueden ser almacenados dentro del atributo value del tipo DV\_BOOLEAN;
  \item el uso de DV\_BOOLEAN, el cual especifica que el tipo se usa para elementos que son datos verdaderamente booleanos, se conserva.
\end{enumerate}


\subsection{Automatización}

Con el fin de comparar los tiempos de creación manual y automática, se llevó a cabo un experimento en el cual se comparó el tiempo requerido para la creación manual y automática del arquetipo de openEHR de recurso Flag de FHIR. Para la creación manual se utilizó el editor ADL WORKbench \cite{ADLWORKbench}. Para la creación automática se implementó el proceso (ver Figura 1) en el lenguaje Python, el cual retorna arquetipos con estructuras equivalentes a los recursos. Los resultados del experimento (ver Cuadro \ref{table:automation}) muestran una reducción significativa en el tiempo de creación usando el proceso automático.

\begin{table}
  \caption{Automatización}
  \label{table:automation}
  \begin{tabular}{l l}
    \hline
    Creación Manual	& Creación Automática \\
    \hline
    Aprox. 30 minutos	& Aprox. 180 milisegundos \\
    \hline
  \end{tabular}
\end{table}

