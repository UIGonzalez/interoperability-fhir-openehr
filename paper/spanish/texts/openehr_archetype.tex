\subsubsection{Arquetipos}

Cada arquetipo es un conjunto de restricciones en el modelo de referencia, definiendo un contenido de dominio \cite{openEHRArchitecture}. Las restricciones definen configuraciones de instancias del modelo de referencia consideradas conformes al arquetipo. Por ejemplo, ciertas configuraciones de las clases PARTY, ADDRESS, CLUSTER y ELEMENT pueden definirse por un arquetipo Person como estructuras permitidas para personas con identidad, contactos y direcciones \cite{openEHRAOM}.

Los arquetipos pueden tener relaciones de especialización o composición. Los arquetipos especializados son creados restringiendo aún más las restricciones existentes de otros arquetipos. Los arquetipos compuestos son definidos a partir de otros arquetipos \cite{openEHRArchitecture}.

OpenEHR incluye un mecanismo de rutas. Estas rutas pueden usarse para referenciar a cualquier dato dentro de un arquetipo \cite{openEHRArchitecture}.

Los arquetipos proveen una forma de definir el significado de los datos, y de conectar los datos a terminologías conocidas como SNOMED CT, LOINC, ICPC, ICDx y muchas otras terminologías y vocabularios usados en salud \cite{openEHRArchitecture}.
