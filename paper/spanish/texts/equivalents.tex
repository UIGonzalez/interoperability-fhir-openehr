\subsection{Equivalencias entre tipos de datos}

La definición de un elemento en un recurso FHIR incluye tipo de dato \cite{FHIRElement}. Por lo tanto, un prerequisito es encontrar equivalencias entre tipos de datos FHIR y tipos de datos openEHR. Las equivalencias permitirán crear arquetipos openEHR a partir de definiciones de recursos FHIR.

La equivalencia entre un tipo de dato primitivo de FHIR y un tipo de dato de openEHR existe si se cumplen las condiciones de:

\begin{enumerate}
  \item el dominio de valores del tipo de dato de FHIR puede ser almacenado dentro de uno de los atributos del tipo de dato de openEHR;
  \item el uso del tipo de dato de FHIR es el mismo que el uso del tipo de dato de openEHR.
\end{enumerate}

Las equivalencias se encuentra al realizar una revisión manual exhaustiva del conjunto de tipos de datos primitivos de FHIR \cite{FHIRDataTypes} y del Modelo de Información de Tipos de Datos de openEHR \cite{openEHRDataTypes}.

Como ejemplo, el tipo de FHIR boolean y el tipo de openEHR DV\_BOOLEAN reúnen ambas condiciones, por lo tanto, son equivalentes:
\begin{enumerate}
  \item los valores true y false del dominio de valores del tipo boolean pueden ser almacenados dentro del atributo value del tipo DV\_BOOLEAN;
  \item el uso de DV\_BOOLEAN, el cual especifica que el tipo se usa para elementos que son datos verdaderamente booleanos, es el mismo uso que tiene el tipo de dato boolean.
\end{enumerate}

Los tipos de datos complejos de FHIR al estar compuestos por elementos con tipos de datos complejos o tipos de datos primitivos no requieren una equivalencia a nivel de tipo de dato.
