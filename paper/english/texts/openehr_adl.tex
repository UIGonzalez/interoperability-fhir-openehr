\subsubsection{Archetype Definition Language}

Archetypes are expressed in archetype definition language \cite{openEHRADL} (ADL). ADL uses three syntaxes: cADL (restriction form of ADL), ODIN (object data instance notation) and a version of first order predicate logic (FOPL).

cADL constraints are written in a block structured style, similar to the block structured programming languages like C. Each block is introduced through an open\-EHR information model identifier. Identifiers toggle between known type names as object blocks or object nodes and known attribute type names as attribute blocks or attribute nodes. The use of object nodes allows the formation of archetype paths that can be used to make object nodes unambiguously within the archetype itself \cite{openEHRADL}.

cADL syntax is used to express archetype definition, meanwhile ODIN syntax is used to express data appearing in the idiom sections, description, terminology and historical revision of an archetype.

Currently, there are two existing main versions: `ADL 1.4', the original version, and `ADL 2', a more modern version, which is slowly being adopted.
