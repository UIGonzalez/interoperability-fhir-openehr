\section{Introduction}

Care coordination involves sharing information among all of the participants concerned with a patient's care to accomplish a better service \cite{CareCoordination}. Nevertheless, this information is disperse among different sites, which hampers healthcare continuity \cite{Indarte11}.

A crucial factor for proper continuity of healthcare is the interoperability among information systems that support the healthcare process through standards \cite{OPS16}. To ensure that the information exchanged can be understood correctly, processed and utilized in an effective manner, clinical concepts and other domain concepts must be standardized using terminologies (LOINC, SNOMED – CT, UCM, etc.) \cite{ISO20514}.

In the field of healthcare, one of the most promising standards is Fast Healthcare Interoperability Resources (FHIR) developed by Health Level 7 (HL7). FHIR combines the functionality of HL7 v2, v3 and CDA with web  standards (XML, JSON, HTTP, OAuth, etc.) \cite{FHIR}. FHIR is based on resources that are basic components for all exchanges. The resources describe clinical and administrative information. In addition to defining resources, FHIR specifies a group of interfaces, which are used by the systems to share information. Another wide use standard is openEHR published by openEHR foundation. OpenEHR specifies a platform to build EHR systems \cite{openEHR}. OpenEHR is based on the modeling of two levels that distinguishes a reference model and archetypes \cite{Bale00}. Reference model is a stable information model that defines the logical structure of the EHR. Archetypes are the formal definitions of clinical concepts constructed on the reference model. One of the purposes of the archetypes is to ensure interoperability.

Some health systems seek to share information using openEHR data repositories through FHIR interfaces \cite{Lopez16}. For this, it is necessary to find equivalences among shared data, FHIR resources, and the domain model of data repositories, openEHR archetypes. Equivalences will enable, through instances of FHIR resources, to generate data in openEHR repositories, and vice-versa.

The goal of this work is to describe an automated process to create openEHR integration archetypes out of FHIR resource definitions using equivalences between open\-EHR and FHIR data types. These equivalences are found manually following the methodology presented as part of the work. The integration archetypes created will aid data exchange between instances of openEHR archetypes and instances of FHIR resources. The remaining of this work is organized in 4 sections. Section 2 presents the main characteristics of FHIR and openEHR, and the architecture of model data conversion. Section 3 describes the creation of openEHR integration archetypes out of FHIR resources using the proposed methodology. Section 4 shows the equivalences between FHIR and openEHR data types, an openEHR integration archetype created with the proposed methodology, and the comparison between the automatic creation and the manual one. Finally, section 5 presents the conclusions of the work.
